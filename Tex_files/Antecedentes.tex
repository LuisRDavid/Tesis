\section{Antecedentes}
\subsection{Aguas residuales}
Las aguas residuales pueden estar constituidas por diversos constituyentes; dentro de los cuales se destacan los físicos, químicos y biológicos~\citep{crites2000}. Es importante caracterizar los distintos tipos de aguas residuales antes de comenzar con algún proceso para la remoción de contaminantes.\par
Antes de ser vertidas en algún cuerpo de agua o suelo, estas deben ser acondicionadas de acuerdo con la normalidad presente en cada país. La misión de estas normativas es mantener una estabilidad en los diferentes ecosistemas, así como el de reducir el número de afecciones a la salud de la población en general~\citep{lazcano2016}. \par
Las aguas residuales son todas aquellas que, una vez son desechadas por cualquier actividad humana o provenientes de precipitaciones, son vertidas a un sistema de alcantarillado para su posterior tratamiento, o en los casos más comunes, son liberadas directamente en algún cuerpo de agua o sobre una superficie de terreno cualquiera. Según sea el caso de uso que recibe el agua es como se clasifica, siendo los principales: aguas residuales domésticas, aguas residuales industriales, aguas pluviales, aguas residuales de origen pecuario y agrícola; y por ultimo las aguas residuales de origen minero-metalúrgico~\citep{lazcano2016,marcos2007}.
\subsubsection{Características Físicas}
\subsubsection*{Sólidos}
Uno de los principales componentes físicos presentes en las aguas residuales son los materiales sólidos dispersos por todo el afluente. El tamaño de estas partículas puede variar desde cabellos hasta materiales coloidales. 
\subsection{Lodos Activados}
Dentro de los procesos basados en cultivo de microorganismos en suspensión, uno de los más importantes, y a su vez mas utilizados, es el que involucra la utilización de lodos activados como agentes reductores de la carga orgánica presente en el afluente a tratar.
\subsubsection{Organismos presentes en los lodos activados}
\subsubsection{Flóculos}
\subsubsection{Bulking filamentoso}
