\section{Antecedentes}
\subsection{Aguas residuales}
Las aguas residuales pueden estar constituidas por diversos constituyentes; dentro de los cuales se destacan los físicos, químicos y biológicos~\citep{crites2000}. Es importante caracterizar los distintos tipos de aguas residuales antes de comenzar con algún proceso para la remoción de contaminantes.\par
Las aguas residuales son todas aquellas que, una vez son desechadas por cualquier actividad humana o provenientes de precipitaciones, son vertidas a un sistema de alcantarillado para su posterior tratamiento, o en los casos más comunes, son liberadas directamente en algún cuerpo de agua o sobre una superficie de terreno cualquiera. Según sea el caso de uso que recibe el agua es como se clasifica, siendo los principales: aguas residuales domésticas, aguas residuales industriales, aguas pluviales, aguas residuales de origen pecuario y agrícola; y por ultimo las aguas residuales de origen minero-metalúrgico.
Antes de ser vertidas en algún cuerpo de agua o suelo, estas deben ser acondicionadas de acuerdo con la normalidad presente en cada país. La misión de estas normativas es mantener una estabilidad en los diferentes ecosistemas, así como el de reducir el número de afecciones a la salud de la población en general~\citep{lazcano2016,martinez1999}.\par
Cabe destacar a este tema que, en la mayoría de países subdesarrollados, la aplicación de estas normativas rara vez se cumplen, resultando en problemas ambientales y de salud graves. La aparición de nuevas industrias locales artesanales y fabricas clandestinas no reguladas provocan un aumento en la cantidad de contaminantes disueltos, entre los cuales, gran parte son metales pesados y/o compuestos de difícil degradación~\cite{metcalf2003}.
Este problema de exacerba cuando no se cuentan con sistemas de tratamiento para las aguas negras generadas por la población, contaminando las distintas fuentes de agua potable de la cuenca en cuestión.\par
\subsubsection*{Aguas residuales domésticas}
Esta categoría se encuentra conformada por todo aquel flujo de agua proveniente de los hogares. Entre los principales constituyentes se incluyen heces y orina de la población; desechos de mascotas, residuos orgánicos producidos por actividades culinarias, desechos de lavandería.
\subsubsection*{Aguas residuales municipales}
Este tipo de aguas provienen de la mezcla de los \glspl{efluente} domésticos, de las distintas actividades realizadas en las áreas urbanas (oficinas, tiendas, centros comerciales, restaurantes, actividades recreativas, etc.) y de las pequeñas industrias locales, las cuales aumentan la cantidad de contaminantes y sustancias indeseadas que dificultan su tratamiento mediante sistemas convencionales aplicados a pequeñas comunidades~\citep{lazcano2016}.
\subsubsection*{Aguas residuales industriales}
Este tipo de aguas provienen de las grandes industrias, a diferencia de las anteriores, estas se caracterizan por estar fuera de las zonas pobladas y debido a su alto contenido en partículas recalcitrantes, estas deben de recibir un tratamiento previo a ser vertidas a los sistemas de alcantarillado público. generalmente cuentan con un número elevado de metales pesados, pH extremo, altos niveles de materia orgánica, solventes y sustancias tóxicas~\citep{lazcano2016}.
\subsubsection*{Aguas residuales agropecuarias o agroindustriales}
Son todos aquellos flujos de agua provenientes de cualquier actividad agrícola y pecuaria. Se encuentran constituidas por una gran cantidad de materia orgánica proveniente del estiércol y purines de los animales, residuos derivados del uso de pesticidas, fertilizantes y residuos farmacéuticos de uso veterinario~\citep{lazcano2016}.
\subsubsection*{Aguas residuales de orígen minero-metalúrgico}
Los efluentes provenientes de la actividad minera son considerados los más tóxicos debido a su alto contenido en metales pesados como el plomo, mercurio, cadmio y zinc; ademas de metaloides antimonio y el arsénico. En países donde la mayor parte de las regulaciones son ignoradas o no son tan estrictas, la cantidad de estos elementos tóxicos supera con creces los límites máximos permitidos, dificultando aún más el tratamiento por medios convencionales. Debido a su alto número en compuestos abióticos, es necesario que este tipo de afluentes reciban un tratamiento anterior a la entrada de cualquier sistema de tratamiento biológico~\citep{lazcano2016}.
\subsubsection*{Aguas pluviales}
Aquellas aguas provenientes de las precipitaciones que terminan en las alcantarillas logran disminuir la carga orgánica que hay en el desagüe, sin embargo, el cambio en las concentraciones produce variaciones en las características fisicoquímicas del agua. Otro aspecto que se debe tomar en cuenta al momento de diseñar un sistema de alcantarillado y de tratamiento es el aumento en los caudales durante el temporal de lluvia~\citep{lazcano2016}.
\subsection{Características de las aguas residuales}
El 
\subsubsection{Características Físicas}
\subsubsection*{Sólidos}
Uno de los principales componentes físicos presentes en las aguas residuales son los materiales sólidos dispersos por todo el afluente. El tamaño de estas partículas puede variar desde cabellos hasta materiales coloidales. 
\subsubsection{Características Químicas}
\subsection{Muestreo}
\subsection{Tratamiento de aguas residuales}

\subsubsection{Tratamientos biológicos}
El uso de organismos vivos con el fin de reducir la cantidad de materia orgánica presente en las aguas negras se remonta a finales del siglo 19 y principios del 20 en Inglaterra,···\par
A consecuencia de el desarrollo de las grandes ciudades, el uso de sistemas biológicos fue adquiriendo más usos a parte de la remoción de materia orgánica. Entre estos nuevos usos se destacan la nitrificación de aguas con alto contenido de nitrógeno amoniacal, la desnitrificación 
\subsubsection*{Procesos biológicos de cultivo en suspensión}
\subsubsection*{Procesos biológicos de soporte sólido}
\subsection{Lodos Activados}
Dentro de los procesos basados en cultivo de microorganismos en suspensión, uno de los más importantes, y a su vez mas utilizados, es el que involucra la utilización de lodos activados como agentes reductores de la carga orgánica presente en el afluente a tratar.
\subsubsection{Organismos presentes en los lodos activados}
\subsubsection{Flóculos}
\subsubsection{Bulking filamentoso}
