\section{Antecedentes}
\subsection{Aguas residuales}
Las aguas residuales pueden estar constituidas por diversos constituyentes; dentro de los cuales se destacan los físicos, químicos y biológicos~\citep{crites}. Es importante caracterizar los distintos tipos de aguas residuales antes de comenzar con algún proceso para la remoción de contaminantes.
\subsubsection{Características Físicas}
\subsubsection*{Sólidos}
Uno de los principales componentes físicos presentes en las aguas residuales son los materiales sólidos dispersos por todo el afluente. El tamaño de estas partículas puede variar desde cabellos hasta materiales coloidales. 
\subsection{Lodos Activados}
Dentro de los procesos basados en cultivo de microorganismos en suspensión, uno de los más importantes, y a su vez mas utilizados, es el que involucra la utilización de lodos activados como agentes reductores de la carga orgánica presente en el afluente a tratar.
\subsubsection{Organismos presentes en los lodos activados}
\subsubsection{Flóculos}
\subsubsection{Bulking filamentoso}
