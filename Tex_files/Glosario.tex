% Dual entries
\newdualentry{DBO} % label
	{DBO$_{5}$}            % abbreviation
	{Demanda Bioquímica de Oxígeno}  % long form
	{Es una estimación de la cantidad de oxígeno que se requiere una población microbiana heterogénea para oxidar la materia orgánica de una muestra de agua en un periodo de 5 días} % description

\newdualentry{DQO}
	{DQO}
	{Demanda Química de Oxígeno}
	{La concentración de masa de oxígeno es equivalente a la cantidad de dicromato consumida por la materia disuelta y suspendida, cuando una muestra de agua es tratada con este oxidante bajo condiciones definidas.
		\begin{center}
			\emph{1 mol de dicromato (Cr\textsubscript{2}O\textsubscript{7}\textsuperscript{2-}) es equivalente a 3 moles de Oxígeno}
		\end{center}
	}

\newdualentry{SSV}
	{SSV}
	{Sólidos Suspendidos Volátiles}
	{Son aquellos sólidos suspendidos que se volatilizan en la calcinación a 550 °C $\pm$ 50 °C}

\newdualentry{SST}
	{SST}
	{Sólidos Suspendidos Totales}
	{Es el material constituido por los sólidos sedimentables, los sólidos suspendidos y coloidales que son retenidos por un filtro de fibra de vidrio con poro de 1.5 $\mu$m secado y llevado a \gls{masacons} a una temperatura de 105 °C $\pm$ 2 °C}

\newdualentry{VUO}
	{VUO}
	{Velocidad de Utilización de Oxígeno}
	{}

\newdualentry{PTAR}
	{PTAR}
	{Planta de Tratamiento de Aguas Residuales}
	{}

\newdualentry{pH}
	{pH}
	{Potencial Hidrógeno}
	{El pH se define en términos de la actividad relativa de los iones de hidrógeno en la disolución:
		$$pH\ =\ -\log a_{H}\ =\ -\log (\frac{m_{H}\gamma_{H}}{m^{0}})$$
	Donde a$_{H}$ es la actividad relativa del ión hidrógeno (en base molal); $\gamma_{H}$ es el coeficiente de actividad molal del ión hidrógeno H$^{+}$ a la molalidad m$_{H}$, y m° es la molalidad estándar. La magnitud pH es considerada como una medida de la actividad de los iones hidrógeno en la disolución}

\newdualentry{mo}
	{MO}
	{materia orgánica}
	{}

\newdualentry{ECA}
	{ECA}
	{Estándares de Calidad Ambiental}
	{Instrumento de gestión ambiental que se establece para medir el estado de la calidad del ambiente en el territorio nacional. El ECA establece los niveles de concentración de elementos o sustancias presentes en el ambiente que no representan riesgos para la salud y el ambiente}

\newdualentry{RHA}
	{RHA}
	{Región Hidrológico - Administrativa}
	{Las RHA están conformadas en función de sus características morfológicas, orográficas e hidrológicas; en ellas se considera a las cuencas hidrológicas como las unidades básicas de gestión de los recursos hídricos.}
	
\newdualentry{NOM}
	{NOM}
	{Norma Oficial Mexicana}
	{Son regulaciones técnicas de carácter obligatorio que establecen especificaciones y procedimientos para garantizar que los productos, procesos y servicios cumplan con requisitos mínimos de información, seguridad, calidad, entre otros.}

% Individual entries
% Glossary entries
\newglossaryentry{masacons}
{
	name={masa constante},
	description={Es la masa que se registra cuando el material ha sido calentado, enfriado y pesado, y que en dos ciclos completos consecutivos presenta una diferencia de $\leq$ 0.0005 g}
}

\newglossaryentry{efluente}
{
	name={efluente},
	description={agua que sale de un recipiente, o un estanque, o una planta de tratamiento o de cualquiera de sus secciones}
}

\newglossaryentry{afluente}
{
	name=afluente,
	description={Es el agua que entra a un depósito, estanque, planta de tratamiento, o alguna de sus secciones}
}

\newglossaryentry{floculos}
{
	name={flóculo biológico},
	description={Agregados de materia inerte, orgánica, microorganismos, etc., presentes en olos sistemas de lodos activados y que intervienen en la estabilización de la \gls{mo}},
	plural={flóculos biológicos}
}


% Acronym entries
\newacronym{INEGI}{INEGI}{Instituto Nacional de Estadística y Geogafía}

\newacronym{CEA}{CEA}{Comisión Estatal del Agua de Jalisco}

\newacronym{CONAGUA}{Conagua}{Comisión Nacional del Agua}