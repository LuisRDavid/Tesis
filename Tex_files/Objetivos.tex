\section{Objetivos}
\subsection{Objetivo general}
Establecer los parámetros cinéticos de crecimiento, degradación de sustrato, producción de biomasa y consumo de oxígeno óptimos para la remoción de contaminantes que permiten el diseño de sistemas más eficientes y la reducción de los costos de operación, empleando distintas fuentes de alimentación (aguas sintéticas y aguas crudas) a escala de laboratorio utilizando lodos activados.
\subsection{Objetivos particulares}
	\begin{enumerate}
		\item Calcular las constantes de crecimiento microbiano de manera experimental de lodos provenientes de una planta de tratamiento en función
		\item Comparar las diferencias que se generan empleando agua residual de constituyentes conocidos frente a un afluente real.
		\item Simular el proceso de remoción de contaminantes utilizando las herramientas presentes en el programa MATLAB® y las constantes que se generan en el proceso.
	\end{enumerate}
