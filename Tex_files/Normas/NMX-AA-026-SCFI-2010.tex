\documentclass[spanish,12pt,letterpaper,titlepage]{article}
\usepackage[left=2cm,right=2cm,top=3cm,bottom=3cm]{geometry}
\usepackage[T1]{fontenc}
\usepackage[utf8x]{inputenc}
\usepackage{hyphenat}
\usepackage{times}
\usepackage{babel}
\usepackage{fancyhdr}
\usepackage{setspace}
\usepackage{titlesec}
\usepackage{amsthm}
\usepackage{fixltx2e}
\usepackage[table,xcdraw]{xcolor}
\singlespacing
\renewcommand*\familydefault{\sfdefault}

\pagestyle{fancy}
\fancyhf{}
\fancyhead[R]{\hfill\large \textbf{Medición de nitrógeno en aguas. Norma NMX-AA-026-SCFI-2010}}
\fancyfoot[R]{}
\renewcommand{\headrulewidth}{0pt}
\renewcommand{\footrulewidth}{0pt}

\titleformat*{\section}{\fontsize{13pt}{0pt}\selectfont\textnormal}
\titleformat*{\subsection}{\fontsize{12pt}{0pt}\selectfont\textnormal}
\titleformat*{\subsubsection}{\fontsize{12pt}{0pt}\selectfont\textnormal}

\theoremstyle{definition}
\newtheorem{teor}{NOTA}

\begin{document}
	\pagestyle{fancy}
	\Large{\textbf{Procedimiento}}
	\normalsize
	\section{Selección del volumen de muestra, véase el cuadro \ref{tab:A1}}\label{A.4.1}
	\begin{table}[!h]
		\centering
		\caption{}
		\label{tab:A1}
		\begin{tabular}{|c|c|}
			\hline
			\rowcolor[HTML]{C0C0C0} 
			Concentración de masa de nitrógeno orgánico en la muestra mg/L & Tamaño de la muestra mL \\ \hline
			4 -- 40                                                        & 50                      \\ \hline
			8 -- 80                                                        & 25                      \\ \hline
			20 -- 200                                                      & 10                      \\ \hline
			40 -- 400                                                      & 5                       \\ \hline
		\end{tabular}
	\end{table}
	\section{Remoción del nitrógeno amoniacal}\label{A.4.2}
	\subsection{En un recipiente de 100 mL colocar 50 mL de muestra o una alícuota apropiada diluida a 50 mL con agua. Añadir 3 mL del buffer de boratos y ajustar el pH a 9,5 con la disolución de hidróxido de sodio 6 mol/L.}\label{A.4.2.1}
	\subsection{Cuantitativamente transferir la disolución obtenida en \ref{A.4.2.1}, a un matraz Kjeldahl de 100 mL. Colocar el matraz en el equipo micro Kjeldahl y permitir que se evaporen aproximadamente 30 mL, en este momento iniciar la destilación como se indica en \ref{A.4.4}.}\label{A.4.2.2}
	\subsection{En caso de no requerir la concentración de masa del Nitrógeno amoniacal, proceder como se indica en \ref{A.4.3}.}\label{A.4.2.3}.
	\section{Digestión}\label{A.4.3}
	\subsection{Cuidadosamente añadir 10 mL de reactivo de digestión al matraz Kjeldahl que contiene la muestra. Añadir algunas perlas de ebullición y colocarlo en el equipo de digestión.}\label{A.4.3.1}
	\subsection{Calentar la disolución obtenida en \ref{A.4.3.1} hasta que se vuelva transparente y se observe la formación abundante de humos ligeramente verdes.}\label{A.4.3.2}
	\subsection{Aumentar el calentamiento al máximo permitido por el equipo y digerir por 30 min más.}\label{A.4.3.3}
	\subsection{Cuantitativamente transferir el contenido del matraz Kjeldahl al equipo de destilación, cuidando que el volumen total transferido no exceda de 30 mL.}\label{A.4.3.4}
	\subsection{Añadir 10 mL de la disolución hidróxido-tiosulfato de sodio y colocar en el destilador, proseguir como se indica en \ref{A.4.4}.}\label{A.4.3.5}.
	\section{Destilación}\label{A.4.4}
	\subsection{Regular la velocidad de destilación para prevenir pérdidas.}\label{A.4.4.1}
	\subsection{Conectar el matraz Kjeldahl al condensador, destilar la muestra cuidando que la temperatura del condensador no pase de 302 K (29 °C).}\label{A.4.4.2}
	\subsection{Recolectar el condensado en un recipiente que contenga 10 mL de la disolución indicadora de ácido bórico, sumergiendo la punta del condensador o una extensión del mismo por debajo de la superficie del líquido.}\label{A.4.4.3}
	\subsection{Retirar el matraz colector y titular con disolución de ácido sulfúrico 0,05 mol/L hasta que el indicador en la disolución vire de verde esmeralda a morado. Registrar el volumen gastado de ácido como volumen C.}\label{A.4.4.4}
	\subsection{Permitir que continúe la destilación por 1 min o 2 min más para que el sistema se limpie.}\label{A.4.4.5}
\end{document}