\Large{\textbf{Procedimiento}}
\normalsize

\begin{itemize}
	\item Selección del volumen de muestra, véase el cuadro \ref{tab:A1a26} \label{A.4.1a26}
	\begin{table}[!h]
		\centering
		\caption{}
		\label{tab:A1a26}
		\begin{tabular}{|c|c|}
			\hline
			\rowcolor[HTML]{C0C0C0} 
			Concentración de masa de nitrógeno orgánico en la muestra mg/L & Tamaño de la muestra mL \\ \hline
			4 -- 40                                                        & 50                      \\ \hline
			8 -- 80                                                        & 25                      \\ \hline
			20 -- 200                                                      & 10                      \\ \hline
			40 -- 400                                                      & 5                       \\ \hline
		\end{tabular}
	\end{table}
	\item Remoción del nitrógeno amoniacal\label{A.4.2a26}
	\subitem En un recipiente de 100 mL colocar 50 mL de muestra o una alícuota apropiada diluida a 50 mL con agua. Añadir 3 mL del buffer de boratos y ajustar el pH a 9,5 con la disolución de hidróxido de sodio 6 mol/L.\label{A.4.2.1a26}
	\subitem Cuantitativamente transferir la disolución obtenida en \ref{A.4.2.1a26}, a un matraz Kjeldahl de 100 mL. Colocar el matraz en el equipo micro Kjeldahl y permitir que se evaporen aproximadamente 30 mL, en este momento iniciar la destilación como se indica en \ref{A.4.4a26}.\label{A.4.2.2a26}
	\subitem En caso de no requerir la concentración de masa del Nitrógeno amoniacal, proceder como se indica en \ref{A.4.3a26}.\label{A.4.2.3a26}.
	\item Digestión\label{A.4.3a26}
	\subitem Cuidadosamente añadir 10 mL de reactivo de digestión al matraz Kjeldahl que contiene la muestra. Añadir algunas perlas de ebullición y colocarlo en el equipo de digestión.\label{A.4.3.1a26}
	\subitem Calentar la disolución obtenida en \ref{A.4.3.1a26} hasta que se vuelva transparente y se observe la formación abundante de humos ligeramente verdes.\label{A.4.3.2a26}
	\subitem Aumentar el calentamiento al máximo permitido por el equipo y digerir por 30 min más.\label{A.4.3.3a26}
	\subitem Cuantitativamente transferir el contenido del matraz Kjeldahl al equipo de destilación, cuidando que el volumen total transferido no exceda de 30 mL.\label{A.4.3.4a26}
	\subitem Añadir 10 mL de la disolución hidróxido-tiosulfato de sodio y colocar en el destilador, proseguir como se indica en \ref{A.4.4a26}.\label{A.4.3.5a26}.
	\item Destilación\label{A.4.4a26}
	\subitem Regular la velocidad de destilación para prevenir pérdidas.\label{A.4.4.1a26}
	\subitem Conectar el matraz Kjeldahl al condensador, destilar la muestra cuidando que la temperatura del condensador no pase de 302 K (29 °C).\label{A.4.4.2a26}
	\subitem Recolectar el condensado en un recipiente que contenga 10 mL de la disolución indicadora de ácido bórico, sumergiendo la punta del condensador o una extensión del mismo por debajo de la superficie del líquido.\label{A.4.4.3a26}
	\subitem Retirar el matraz colector y titular con disolución de ácido sulfúrico 0,05 mol/L hasta que el indicador en la disolución vire de verde esmeralda a morado. Registrar el volumen gastado de ácido como volumen C.\label{A.4.4.4a26}
	\subitem Permitir que continúe la destilación por 1 min o 2 min más para que el sistema se limpie.\label{A.4.4.5a26}
\end{itemize}