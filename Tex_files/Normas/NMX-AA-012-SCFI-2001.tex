\Large{\textbf{Procedimiento}}
\normalsize
\begin{itemize}
	\item Método electrométrico
	\\Posterior a la calibración del instrumento proceder a hacer la medición de la(s) muestra(s) siguiendo el procedimiento descrito a continuación.}\label{9.1A12}
	\subitem Introducir el electrodo previamente lavado con agua a la muestra.\label{9.1.1A12}
	\subitem Agitar uniformemente y leer directamente del instrumento la concentración de oxígeno. \label{9.1.2A12}
	\subitem Método Yodométrico \label{9.2A12}
	\subitem Determinación de OD\label{9.2.1A12}
	\subitem Para fijar el oxígeno, adicionar a la botella tipo Winkler que contiene la muestra (300 mL), 2 mL de sulfato manganoso.\label{9.2.2A12}
	\subitem Agregar 2 mL de la disolución alcalina de yoduro-azida.\label{9.2.3A12}
	\subitem Tapar la botella tipo Winkler, agitar vigorosamente y dejar sedimentar el precipitado.\label{9.2.4A12}
	\subitem Añadir 2 mL de ácido sulfúrico concentrado, volver a tapar y mezclar por inversión hasta completa disolución del precipitado.\label{9.2.5A12}
	\subitem Titular 100 mL de la muestra con la disolución estándar de tiosulfato de sodio 0.025 M agregando el almidón hasta el final de la titulación, cuando se alcance un color amarillo pálido. Continuar hasta la primera desaparición del color azul.\label{9.2.6A12}
\end{itemize}