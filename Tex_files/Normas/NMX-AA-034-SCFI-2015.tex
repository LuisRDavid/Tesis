\documentclass[spanish,12pt,letterpaper,titlepage]{article}
\usepackage[left=2cm,right=2cm,top=3cm,bottom=3cm]{geometry}
\usepackage[T1]{fontenc}
\usepackage[utf8x]{inputenc}
\usepackage{hyphenat}
\usepackage{times}
\usepackage{babel}
\usepackage{fancyhdr}
\usepackage{setspace}
\usepackage{titlesec}
\usepackage{amsthm}
\usepackage{fixltx2e}
\singlespacing
\renewcommand*\familydefault{\sfdefault}

\pagestyle{fancy}
\fancyhf{}
\fancyhead[R]{\hfill\large \textbf{Medición de Sólidos. Norma NMX-AA-034-SCFI-2015}}
\fancyfoot[R]{}
\renewcommand{\headrulewidth}{0pt}
\renewcommand{\footrulewidth}{0pt}

\titleformat*{\section}{\fontsize{13pt}{0pt}\selectfont\textnormal}
\titleformat*{\subsection}{\fontsize{12pt}{0pt}\selectfont\textnormal}
\titleformat*{\subsubsection}{\fontsize{12pt}{0pt}\selectfont\textnormal}

\theoremstyle{definition}
\newtheorem{teor}{NOTA}

\begin{document}
	\pagestyle{fancy}
	\noindent\Large{\textbf{Procedimiento}}
	\normalsize
	\section{Preparación de cápsulas}\label{9.1}
	\subsection{Introducir las cápsulas al horno a una temperatura de 105 °C \unichar{"00B1} 2 °C, 20 minutos como mínimo. Únicamente en el caso de la medición de sólidos volátiles, las cápsulas posteriormente se introducen a la mufla a una temperatura de 550 °C \unichar{"00B1} 50 °C, durante 20 minutos como mínimo. Después de este tiempo transferirlas al horno.}\label{9.1.1}
	\subsection{Trasladar la cápsula al desecador y dejar enfriar por 20 minutos como mínimo.}\label{9.1.2}
	\begin{teor}
		El manejo de la cápsula durante el análisis, debe realizarse en todo momento con las pinzas.
	\end{teor}
	\subsection{Pesar las cápsulas y repetir el ciclo horno-desecador (véase \ref{9.1.1} y \ref{9.2.2}) hasta obtener una diferencia $\leq$ 0,0005 g en dos pesadas consecutivas. Registrar como m\textsubscript{1} considerando para los cálculos el último valor de la masa.}\label{9.1.3}
	\section{Preparación del dispositivo de filtración y/o soportes de secado.}\label{9.2}
	\subsection{Utilizar filtro de fibra de vidrio que adapte al dispositivo de filtración y/o secado y/o charola de aluminio, con la ayuda de unas pinzas colocarlo con la cara rugosa hacia arriba en el dispositivo de secado y/o filtración.}\label{9.2.1}
	\begin{teor}
		Mojar el filtro con agua para asegurar que se adhiera perfectamente, solo en caso de utilizar crisol Gooch.
	\end{teor}
	\subsection{El soporte de secado con el filtro se introduce al horno a 105 °C ± 2 °C durante 20 minutos como mínimo, después de este tiempo transferirlo a un desecador.} \label{9.2.2}
	\subsection{Pesar el dispositivo de filtración y/o soportes de secado y repetir el ciclo horno-desecador (véase \ref{9.2.2}) hasta obtener una diferencia $\leq$ 0,0005 g en dos pesadas consecutivas. Registrar como m\textsubscript{2}, considerando para los cálculos el último valor de la masa.}\label{9.2.3}
	\section{Preparación de la muestra}\label{9.3}
	\subsection{Las muestras deben estar a temperatura ambiente al realizar el análisis. Agitar las muestras para asegurar la homogenización.}\label{9.3.1}
	\section{Medición de sólidos totales (ST) y sólidos volátiles (STV).}\label{9.4}
	\subsection{Medición de sólidos totales (ST)}\label{9.4.1}
	\subsubsection{Se recomienda seleccionar el volumen de muestra de tal manera que el residuo seco sobre la cápsula se encuentre en un intervalo de masa de 2.5 mg a 200 mg.}\label{9.4.1.1}
	\subsubsection{Transferir la muestra a la cápsula previamente puesta a masa constante (véase \ref{9.1.3}) y evaporar a sequedad en el horno de secado a 105 °C ± 2 °C.}\label{9.4.1.2}
	\subsubsection{En caso de utilizar placa de calentamiento llevar a casi sequedad sin llegar a ebullición de la muestra y posteriormente pasar al horno de secado a 105 °C ± 2 °C para su secado total por una hora.}\label{9.4.1.3}
	\subsubsection{Trasladar la cápsula al desecador y dejar enfriar por 20 minutos como mínimo. Llevar la cápsula a masa constante repitiendo el ciclo horno-desecador (véase \ref{9.1.1} y \ref{9.1.2}), hasta obtener una diferencia $\leq$ 0,000 5 g en dos pesadas consecutivas.}\label{9.4.1.4}
	\subsubsection{Registrar como m\textsubscript{3}, la última masa obtenida.}\label{9.4.1.5}
	\subsection{Medición de sólidos totales volátiles (STV)}\label{9.4.2}
	\subsubsection{Introducir la cápsula conteniendo el residuo (véase \ref{9.4.1.1}) a la mufla a 550 °C $\leq$ 50 °C durante 15 minutos a 20 minutos, transferir la cápsula al horno a 105 °C  2 °C, 20 minutos como mínimo. Trasladar la cápsula siguiendo el punto 9.4.1.4, y registre el valor como m\textsubscript{4}.}\label{9.4.2.1}
	\section{Sólidos disueltos totales(SDT)}\label{9.5}
	\subsection{Para la medición de los sólidos disueltos totales proceda con los cálculos; si no se poseen tales datos, pasar al punto \ref{9.5.2}.}\label{9.5.1}
	\subsection{En la cápsula llevada previamente a masa constante m\textsubscript{1}, filtrar una alícuota de la muestra a través de un filtro de fibra de vidrio en el crisol o dispositivo de filtrado. Verter la alícuota en una cápsula preparada (véase \ref{9.1}) y evaporar a sequedad en el horno de secado a 105 °C ± 2 °C o evaporar casi a sequedad sin llegar a ebullición de la muestra, en una parrilla de calentamiento.\vspace{12pt}\\Introducir al horno a 105 °C ± 2 °C la cápsula con la muestra, durante al menos 1 hora. Pasar la cápsula al desecador para llevar a masa constante (véase \ref{9.4.1.4}). Registrar como m\textsubscript{5}.}\label{9.5.2}
	\begin{teor}
		Si al cabo de 1 h aún se observa humedad o líquido en la cápsula, continuar secando en el horno.
	\end{teor}
	\section{Medición de sólidos suspendidos totales (SST) y Medición de sólidos suspendidos volátiles (SSV)}\label{9.6}
	\subsection{Medición de sólidos suspendidos volátiles (SST)}\label{9.6.1}
	\subsubsection{Se recomienda seleccionar el volumen de muestra de acuerdo a las características de esta.}\label{9.6.1.1}
	\subsubsection{Homogeneizar la muestra mediante agitación vigorosa del envase, transferir de forma inmediata y en un solo paso un volumen adecuado de muestra a una probeta.}\label{9.6.1.2}
	\subsubsection{Filtrar la muestra:\\
	\\a) A través del filtro colocado en el crisol Gooch (véase \ref{9.2}) o\\ b) A través del filtro que es tomado de la charola de aluminio y colocado en el equipo de filtración con ayuda de unas pinzas (véase \ref{9.2}).
	\vspace{12pt}\\Enjuagar la probeta con el volumen suficiente para arrastrar los sólidos y verter en el filtro.}\label{9.6.1.3}
	\begin{teor}
		Algunos tipos de agua contienen materiales que bloquean los poros del filtro o reducen su diámetro. Esto incrementa el tiempo de filtrado y los resultados se relacionan en función del volumen de la muestra. Si se observa tal bloqueo del filtro, deberá repetirse la medición con menor volumen. Los resultados deberán interpretarse considerando lo anterior.
	\end{teor}
	\subsubsection{Introducir el soporte de secado con el filtro al horno a 105 °C ± 2 °C durante 1 h como mínimo, en caso de usar un soporte de secado diferente al crisol Gooch retirar con cuidado el filtro del equipo de filtrado usando pinzas. Posteriormente llevar a masa constante véase \ref{9.4.1.4} y registrar como m\textsubscript{6} la masa obtenida.}\label{9.6.1.4}
	\subsection{Medición de sólidos suspendidos volátiles (SSV)}\label{9.6.2}
	\subsubsection{Introducir el soporte de secado con el filtro que contiene el residuo m\textsubscript{6} a la mufla a una temperatura de 550 °C ± 50 °C durante 15 min a 20 min.}\label{9.6.2.1}
	\subsubsection{Trasladar el soporte de secado con el filtro al horno a una temperatura de 105 °C ± 2 °C durante 20 min como mínimo.}\label{9.6.2.2}
	\subsubsection{Transferir el soporte de secado con el filtro al desecador y llevar a masa constante (véase \ref{9.4.1.4}.) Registrar como m\textsubscript{7}.}\label{9.6.2.3}
\end{document}