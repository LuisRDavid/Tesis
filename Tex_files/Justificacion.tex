\section{Justficación}
Los datos presentados en \cite{ODS23} muestran que menos de la mitad del agua residual generada es tratada de manera adecuada, sobre todo en la región hidrológica Lerma-Santiago donde se tiene un deficiente control de las zonas de descarga de las redes colectoras, vertiendo la mayoría en cuerpos de agua superficiales generando problemas sanitarios en poblaciones que no cuentan con plantas potabilizadoras (como es el caso de Lagos de Moreno) \citep{CEAJ2015}.\par
Por tal motivo resulta esencial el fomentar la investigación en técnicas de tratamiento de aguas residuales con vistas a reducir la carga de contaminantes de los cuerpos de agua y generar una recarga artificial de los acuíferos, reduciendo el déficit de agua al que se enfrentan actualmente los acuíferos del país y proponiendo un sistema capaz de satisfacer las necesidades de la población a un bajo coste y con mayor eficiencia de trabajo.\par
Actualmente ya se cuenta con varios modelos, siendo los \gls{ASM} los más representativos y respaldados, contando con varios trabajos de investigación que demuestran la concordancia entre los datos reales y las predicciones generadas por el modelo.\par
\cite{Costa2022}, realiza una comparación entre la aplicación del modelo \acrshort{ASM}1 y la \acrshort{PTAR} de Salamanca España, concluyendo que, aunque el modelo realiza predicciones que llegan a acercarse al resultado in-situ, es necesario que se realicen ajustes a los parámetros \textit{\unichar{"00B5}}$_{H,max}$ y $Y_{H}$ con el fin de conseguir una mejor representación y reducción del error promedio entre los valores de simulación y los datos analizados. También destaca que, si bien el modelado matemático no es capar de predecir la variabilidad generada inevitablemente con respecto a las fluctuaciones en las condiciones de operación a las que son sometidas las comunidades bacterianas dentro del sistema, este proceso sirve para comprender las causas de tales variaciones y como afectan al sistema de manera general con el fin de seguir mejorando el modelo, ya que como lo menciona el autor, el modelo original fue publicado hace mas de 30 años y , aunque se han desarrollado modelos que solventan errores u omisiones del primer modelo, estos están aún lejos de tener errores entre el valor simulado y el analizado in-situ.\par
En otro articulo, \cite{Petersen2002}