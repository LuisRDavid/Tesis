\section{Justficación}
Los datos presentados por en \cite{ODS23} muestran que menos de la mitad del agua residual generada es tratada de manera adecuada, sobre todo en la región hidrológica Lerma-Santiago donde se tiene un deficiente control de las zonas de descarga de las redes colectoras, vertiendo la mayoría en cuerpos de agua superficiales generando problemas sanitarios en poblaciones que no cuentan con plantas potabilizadoras (como es el caso de Lagos de Moreno) \citep{CEAJ2015}.\par
Por tal motivo resulta esencial el fomentar la investigación en técnicas de tratamiento de aguas residuales con vistas a reducir la carga de contaminantes de los cuerpos de agua y generar una recarga artificial de los acuíferos, reduciendo el déficit de agua al que se enfrentan actualmente los acuíferos del país y proponiendo un sistema capaz de satisfacer las necesidades de la población a un bajo coste y con mayor eficiencia de trabajo.\par
Actualmente ya se cuenta con varios modelos, siendo los \gls{ASM} los más representativos y respaldados, contando con varios trabajos de investigación que demuestran la concordancia entre los datos reales y las predicciones generadas por el modelo.